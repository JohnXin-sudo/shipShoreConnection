\chapter{绪论}

这是 \shmtuthesis 的示例文档,基本上覆盖了模板中所有格式的设置。建议大家在使用模板之前,除了阅读《\shmtuthesis\ 使用文档》,这个示例文档也最好能看一看。

\section{二级标题}

\subsection{三级标题}

\subsubsection{四级标题}

\zhlipsum[1-2]

\section{脚注}

上海海事大学的校训 “忠信笃敬” ,通常形容言行举止很忠义,值得别人相信,自己做的事也受到别人的尊敬。\footnote{出自《论语·卫灵公》:“言忠信,行笃敬,虽蛮貊之邦,行矣.言不忠信,行不笃敬,虽州里,行乎哉?”}

\section{字体}
{默认字体——宋体:中国高等航海教育发轫于上海,1909年晚清邮传部上海高等实业学堂(南洋公学)船政科开创了我国高等航海教育的先河。1912年成立吴淞商船学校,1933年更名为吴淞商船专科学校。1959年交通部在沪组建上海海运学院。2004年经教育部批准更名为上海海事大学。为更好地服务上海国际航运中心建设和国家航运事业发展,根据上海市高校布局结构调整规划,2008年上海海事大学主体搬迁临港新城(现上海自贸区临港新片区)。2019年学校成功举行110年校庆系列活动。} 

{宋体——\cs{songti}:\songti 上海海事大学是一所以航运、物流、海洋为特色,具有工学、管理学、经济学、法学、文学、理学和艺术学等学科门类的多科性大学。2008年,上海市人民政府与交通运输部签订协议,共建上海海事大学。}

{黑体——\cs{heiti}:\heiti 学校设有3个博士后科研流动站(交通运输工程、电气工程、管理科学与工程),4个一级学科博士点(交通运输工程、管理科学工程、船舶与海洋工程、电气工程),23个二级学科博士点,16个一级学科硕士学位授权点,60个二级学科硕士学位授权点,12个专业学位授权类别,48个本科专业。拥有12个省部级重点研究基地。现有1个国家重点(培育)学科,1个上海市高峰学科,2个上海市高原学科,9个部市级重点学科,工程学科进入ESI全球前1\%,港航物流学科保持全球领先。5个国家级特色专业,1个国家级综合改革试点专业,7个国家级一流本科专业建设点,6个教育部卓越工程师教育培养计划专业,17个上海市本科教育高地。现有2个国家级实验教学示范中心,2个国家级虚拟仿真实验教学示范中心,5个国家级实践教学示范中心,1个全国示范性工程专业学位研究生联合培养基地。设有水上训练中心,拥有万吨级集装箱教学实习船“育锋”轮,4.8万吨散货教学实习船“育明”轮。}

{楷书——\cs{\kaishu}:\kaishu 在2004年教育部本科教学工作水平评估和2006年教育部英语专业教学评估中获得优秀。2018年,年度科技总经费达到3.7亿元,获一批国家级科研项目及部市级以上科技进步奖。}

{仿宋——\cs{fangsong}:\fangsong 实行校院二级管理体制,现设有商船学院、交通运输学院、经济管理学院(设亚洲邮轮学院)、物流工程学院(设中荷机电工程学院)、法学院、信息工程学院、外国语学院、海洋科学与工程学院、文理学院(设马克思主义学院)、徐悲鸿艺术学院、物流科学与工程研究院、上海高级国际航运学院等二级办学部门。在24000余名学生中,有本科生16500余人,各类在校研究生近6000人,留学生近700名。在1200余名专任教师中,有教授160余名,具有博士学位的教师比例约63\%。学校致力于培养国家航运业所需要的各级各类专门人才,已向全国港航企事业单位及政府部门输送了毕业生逾16万,被誉为“高级航运人才的摇篮”。}

{隶书——\cs{lishu}:\lishu 学校2013年成立中国(上海)自贸区供应链研究院和上海高级国际航运学院。中国(上海)自贸区供应链研究院将自贸区建设与供应链研究有机结合,以提升自贸区产业链建设水平,促进自贸区货物贸易向服务贸易的转型发展,同时推动政府监管职能的转变。上海高级国际航运学院采取国际上先进的商学院运作模式,与全球优秀教育机构资源共享,着力打造国内领先、国际知名的航运金融教育品牌,构筑具有影响力的航运高端人才输出基地。}

{幼园——\cs{youyuan}:\youyuan 2008年,上海市教育委员会、上海市城乡建设和交通委员会、上海海事大学、虹口区人民政府等20多家单位共同发起成立上海国际航运研究中心。中心挂靠上海海事大学,是国际航运业发展的研究和咨询机构,为政府和国内外企业与航运机构等提供决策咨询和信息服务,是上海市教委首批建立的“高校知识服务平台”之一。 2014年,市教委将该平台挂牌为“上海市协同创新中心”。

学校与境外100余所姐妹院校建立了校际交流与合作关系,开展教师交流、合作办学、合作科研、学生交换等。与联合国国际海事组织、波罗的海国际航运公会、挪威船级社等国际知名航运组织/机构建立了密切联系。自2010年起开设“国际班”,邀请美国、韩国、波兰、俄罗斯、德国等国家航海院校的学生来校学习“航海技术”“航运管理”等专业。2011年,经教育部批准,学校与加纳中西非地区海事大学合作举办“物流管理”本科教育项目,并开始在非洲招生,这是上海市地方高校第一个颁发中国高校本科文凭的海外办学项目。2012年,学校获教育部批准正式成为“接受中国政府奖学金来华留学生院校”。}

\section{字号}

\begin{itemize}
	\item {\zihao{0} 初号:\cs{zihao{0}}}
	\item {\zihao{-0} 小初:\cs{zihao{-0}}}
	\item {\zihao{1} 一号:\cs{zihao{1}}}
	\item {\zihao{-1} 小一号:\cs{zihao{-1}}}
	\item {\zihao{2} 二号:\cs{zihao{2}}}
	\item {\zihao{-2} 小二号:\cs{zihao{-2}}}
	\item {\zihao{3} 三号:\cs{zihao{3}}}
	\item {\zihao{-3} 小三号:\cs{zihao{-3}}}
	\item {\zihao{4} 四号:\cs{zihao{4}}}
	\item {\zihao{-4} 小四号:\cs{zihao{-4}}}
	\item {\zihao{5} 五号:\cs{zihao{5}}}
	\item {\zihao{-5} 小五号:\cs{zihao{-5}}}
	\item {\zihao{6} 六号:\cs{zihao{6}}}
	\item {\zihao{-6} 小六号:\cs{zihao{-6}}}
	\item {\zihao{7} 七号:\cs{zihao{7}}}
	\item {\zihao{8} 八号:\cs{zihao{8}}}
\end{itemize}
