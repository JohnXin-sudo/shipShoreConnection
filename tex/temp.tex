\chapter{并网相关}

先来了解一下传播并网的整体过程:

1.船舶到港后,由港口工作人员将岸电电缆连接到船侧岸电接受装置。
2.系统运行初始化和自检操作。
3.检查电缆连接情况,检测船岸通讯状况。
4.船岸连接系统正常运行的情况下,开始执行并网程序。
5.并网可以自动运行也可以操作到手动模式,在船上发出并网信号后,岸电开始输出稳定电压
船舶电网电压应略低于岸电电源,并网分两种模式:
1)主动并网模式:岸电控制系统通过船岸连接系统获取船上发电机组,船上电网电压电流频率和相位相关信息
               主动调整自身电压、频率、相位,达到并网标准时就可以发出并网信号,断路器合闸,并网完成。
               接着负载转移程序开始执行,由于并网控制程序有岸电系统操控,负载转移时,需要船舶PMS配合,
               意味着需要对船舶PMS进行改造。负载转移需要系统控制船舶发电机组逐渐降低输出功率,控制船舶
               电网电压略低于岸电,一般为不大于5\%使冲击电流尽可能小,这样就可以使得船舶发电机组的电流
               逐渐减少至0(此时也要防止出现逆功率情况),当船舶发电机组功率小于其额定功率的10\%时船舶
               发电机组退出工作,船舶负载全部转移至岸电。
               缺点:未被实际采用,原因是需要对船舶PMS系统进行改造,会对原有的系统控制造成混乱,改造成本高,
               岸上电源控制系统可以直接操控船舶发电机组,可能数据泄漏,存在网络安全问题。在应用广泛的集中式
               的岸电系统中改变岸电频率容易对其它船只使用岸电造成干扰。
               优点:在分布式和直流输电式岸电系统中,整个并网系统可以被岸电系统进行事实监控,可以控制船岸连接
               系统的每一个部分。
2)被动并网模式:采用广泛,岸电作为稳定电源工作,或者接受船上并网系统的指令输出对应的频率和幅值的电压,并网时,
               一般船载并网控制系统控制,控制船载发电机,使船舶电网电压略低于岸电,然后通过发电机调速系统,
               使船电向岸电同步,达到并网要求后,并网合闸,由于船电略低于岸电,负载开始向岸电转移,当船舶发电机
               组低于其容量的10\%时退出船舶电网。
               
               